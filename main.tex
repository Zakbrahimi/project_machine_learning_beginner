\documentclass{article}
\usepackage{graphicx} % Required for inserting images

\title{TP explications}
\author{Zakaria Brahimi }
\date{October 2025}

\begin{document}

\maketitle

\section{Introduction}
Concernant le TP, j'ai utilisé l'outil de google Colab Notebooks pour pouvoir coder en Python en ligne car je n'ai pas Python sur mon ordinateur. Pour chaque fonction utilisée, j'utilisais la commande help() pour comprendre comment celle-ci fonctionne. Pour ce qui est du déroulement du TP, tout est déja indiqué à chaque étape. Je n'ai élaboré aucun algorithme, j'ai seulement suivi chaque étape.
Par exemple, le TP commence avec la première tache: \begin{it}
    First we need some "linear" data that we can play with. Then preproces the data by splitting between training and testing (80/20) and standardize data.
\end{it}
Ensuite, on déroule tout simplement le TP et on fini avec les différents graphiques. Sinon, je ne sais pas trop quoi rajouter, j'ai surtout crée ce fichier tex car il fallait faire ainsi sur moodle.
\end{document}
